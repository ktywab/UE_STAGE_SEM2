\documentclass{article}
\usepackage[a4paper, left=2.5cm, right=2.5cm, top=2.5cm, bottom=2.5cm]{geometry}
\usepackage{graphicx}
\usepackage{tikz}
\usepackage[french]{babel}
\usepackage{amssymb}
\usepackage{amsmath}
\usepackage{dsfont}
\usepackage{float}
\usepackage{caption}
\usepackage[utf8]{inputenc}
\usepackage{array}
\usepackage{adjustbox}
\usepackage{listings}
\usepackage{hyperref}
\usepackage{graphicx} 
\usepackage[backend=biber,style=numeric,sorting=nyt]{biblatex} % ICI POUR LA BIBLIO
\addbibresource{references.bib} % Nom du fichier .bib % ICI POUR LA BIBLIO
\lstset{
  language=R,                     
  basicstyle=\footnotesize\ttfamily, 
  numbers=left,                   
  numberstyle=\tiny\color{gray},  
  stepnumber=1,                   
  numbersep=5pt,                  
  backgroundcolor=\color{white},  
  showspaces=false,               
  showstringspaces=false,         
  showtabs=false,                
  frame=none,                   
  tabsize=2,                   
  captionpos=b,                   
  breaklines=true,                
  breakatwhitespace=false,        
  title=\lstname,                 
  keywordstyle=\color{blue},      
  commentstyle=\color{purple},    
  stringstyle=\color{red},        
  escapeinside={\%*}{*)},           
  literate={
    {²}{{\textsuperscript{2}}}1
    {⁴}{{\textsuperscript{4}}}1
    {⁶}{{\textsuperscript{6}}}1
    {⁸}{{\textsuperscript{8}}}1
    {€}{{\euro{}}}1
    {é}{{\'e}}1
    {è}{{\`e}}1
    {ê}{{\^{e}}}1
    {ë}{{\¨e}}1
    {É}{{\'E}}1
    {Ê}{{\^{E}}}1
    {û}{{\^{u}}}1
    {ù}{{\`u}}1
    {â}{{\^{a}}}1
    {à}{{\`a}}1
    {á}{{\'a}}1
    {ã}{{\~a}}1
    {Á}{{\'A}}1
    {Â}{{\^{A}}}1
    {Ã}{{\~A}}1
    {ç}{{\c{c}}}1
    {Ç}{{\c{C}}}1
    {õ}{{\~o}}1
    {ó}{{\'o}}1
    {ô}{{\^{o}}}1
    {Õ}{{\~O}}1
    {Ó}{{\'O}}1
    {Ô}{{\^{O}}}1
    {î}{{\^{i}}}1
    {Î}{{\^{I}}}1
    {í}{{\'i}}1
    {Í}{{\~Í}}1
  },
  morekeywords={*,...}
}


\begin{document}



\begin{titlepage}

    \begin{tikzpicture}[remember picture, overlay]
        \node[anchor=north west, inner sep=15pt] at (current page.north west) {\includegraphics[width=10cm]{TG.jpg}};
    \end{tikzpicture}

    \begin{tikzpicture}[remember picture, overlay]
        \node[anchor=north east, inner sep=10pt] at (current page.north east) {\includegraphics[width=10cm]{virus.png}};
    \end{tikzpicture}
   
    \begin{tikzpicture}[remember picture, overlay]
        \node[anchor=south, inner sep=30pt] at (current page.south) {\includegraphics[width=6cm]{ufr.jpeg}};
    \end{tikzpicture}

    
    \vspace*{7cm}


    \begin{center}
       {\Huge Rapport de stage} \par
       \vspace{1cm}
       {\Large Simulation de données de survie dans le cadre d’essais cliniques} \par
       \vspace{1cm}
       {\Huge TAHAR Jimmy} \par
    \end{center}
     
       \vspace{2cm}
       {\Large Tutrice référente : Madame Martin} \par
       \vspace{0.5cm}
       {\Large Enseignant référent : GARDES Laurent} \par

    \vspace{4cm}
    
    \begin{center}
       Du 22 Janvier 2024 au 22 Juillet 2024
    \end{center}

 
\end{titlepage}

\title{\textbf{Remerciements}}
\date{} 
\maketitle

\newpage

% Table des matière
\renewcommand{\contentsname}{Table des matières} 
\tableofcontents
\newpage

\section{Présentation}
\subsubsection{Sous partie}

\section{Développement}

Tu peux ecrire un truc la et tu cites \cite{cours} \cite{einstein1935} \cite{leem} \cite{R} \cite{R} \cite{survivalanalysis} \cite{tg}

\section{Conclusion}

\newpage


\renewcommand{\refname}{Bibliographie}

\addcontentsline{toc}{section}{Bibliographie}

\printbibliography



\newpage


\section*{Annexe}
\addcontentsline{toc}{section}{Annexe} 

\begin{lstlisting}[language=R] %ici tu peux mettre du code R c'est pratique (je t'ai mis une fonction à moi en exemple)

Recrutement_temps <- function(
    t,                # Temps de recrutement
    site,             # Nombre de sites actuellement ouverts
    max_site,         # Nombre maximum de sites autorisés
    tx_site,          # Taux de recrutement par site initial
    variation_tx_site, # Variation du taux de recrutement par site
    temps_fin_ouverture, # Temps à partir du moment où on n'ouvre plus de centre 
    ouverture,       # Nbr de mois ou tout doit statistiquement ouvrir
    boost = 1,        # Temps  à partir du moment où je boost la probabilité du tx_site
    nbr_site_boost = 0  # Nombre boost de site
) {
  
  # Calcul de la probabilité d'ouverture d'un centre
  
  # je fais en sorte que tout s'ouvre environ au bout de "ouverture" mois
  
  a<-(ouverture)*(ouverture+1)/2
  
  proba1 <- seq(ouverture/a, 1/a, length.out = ouverture)
  
  valeur<- 1/a
  
  indice<-0:90
  
  proba2<-valeur*(0.95^indice)
  
  Proba_ouverture_centre <- c(proba1,proba2)
  
  
  # Nombre de sites pouvant ouvrir
  Site_qui_peuvent_ouvrir <- max_site - site 
  
  # Initialisation des compteurs et des dataframes de recrutemen
  z<-0
  r <- 0
  compteur <- 0
  recru <- data.frame(Temps = c(0), Recrutement = c(0), Nombre_site = c(site))
  recru2 <- data.frame(Temps = c(0), Recrutement = c(0), Nombre_site = c(site))
  condition <- TRUE
  
  # Boucle principale de recrutement
  for (i in 1:t) {
    # Ouverture de centres jusqu'à temps_fin_ouverture
    if (i <= temps_fin_ouverture ) {
      a <- rbinom(Site_qui_peuvent_ouvrir, 1, Proba_ouverture_centre[compteur + 1])
      a <- sum(a)
      site <- site + a
      Site_qui_peuvent_ouvrir <- Site_qui_peuvent_ouvrir - a
    }
    
    # Activation du boost si nécessaire
    if (i >= boost & condition == TRUE) {
      condition <- FALSE
    }
    
    # Calcul du taux de recrutement
    taux <- tx_site + runif(1, -variation_tx_site, variation_tx_site)
    
    # Calcul du recrutement
    r <- r + site * taux
    
    # Mise à jour des données de recrutement
    compteur <- compteur + 1
    recru <- rbind(recru, c(compteur, r, site))
    
    # Si le boost est activé, appliquer le boost 
    
    if (condition == FALSE) {
      z<-z+1
      recru2 <- rbind(recru2, c(compteur, r + z*(nbr_site_boost*taux), site))
    } else {
      recru2 <- rbind(recru2, c(compteur, r, site))
    }
  }
  
  # Création du graphique
  graph <- ggplot(data = recru, aes(x = Temps, y = Recrutement)) + 
    geom_line(data = recru, aes(x = Temps, y = Nombre_site)) +
    geom_line(color = "red") +
    geom_line(data = recru2, aes(x = Temps, y = Recrutement), color = "orange") 
  
  
  # Conversion du graphique en interactive avec ggplotly
  graph <- ggplotly(graph)
  
  return(list(plot=graph,recru=recru,recruboost=recru2))
  
  ############################################################# 
  ###########Ecart de MVPA entre filles et garçons#############
  #############################################################
  #Test de Fisher Snedecor (Egalité de variance dans les deux groupes)
  test_F = var.test(data_F$ecart_MVPA,data_M$ecart_MVPA)
  print(test_F)
  #Test de student
  t_test = t.test(ecart_MVPA~gender,data = data,var.equal = TRUE)
  print(t_test)
  #Taille d'effet d de Cohen
  taille_effet = cohen.d(data_M$ecart_MVPA,data_F$ecart_MVPA)
  print(taille_effet)
  ############################################################# 
  ##Les écarts de MVPA entre filles et garçons selon les CA ###
  #############################################################
  
}

\end{lstlisting}

\end{document}